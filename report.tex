\documentclass{whureport}
% =============================================
% Part 1 Edit the info
% =============================================
\usepackage{booktabs}%三线表
\usepackage[breaklinks,colorlinks,linkcolor=black,
citecolor=black,urlcolor=black]{hyperref}% 生成书签
\pagecolor [rgb]{0.9, 0.99, 0.9}%保护视力背景色
\newcommand{\major}{电子信息工程}
\newcommand{\name}{傅宇千}
\newcommand{\stuid}{2018302120169}
\newcommand{\newdate}{\today}
\newcommand{\loc}{None}

\newcommand{\course}{信号与系统}
\newcommand{\tutor}{PU}
\newcommand{\grades}{~~~~~~~}
\newcommand{\newtitle}{Time-Series}
\newcommand{\exptype}{None}
\newcommand{\group}{徐梅隆、张笑}

% =============================================
% Part 1 Main document
% =============================================
\begin{document}
\thispagestyle{empty}
\begin{figure}[h]
  \begin{subfigure}{0.4\linewidth}
    \centerline{\includegraphics[height=3cm,width=3cm]{whulogo.eps}}
  \end{subfigure}
  \hfill
  \begin{subfigure}{.5\linewidth}
    \raggedleft
    \begin{tabular*}{.8\linewidth}{ll}
      专业: & \underline\major   \\
      姓名: & \underline\name    \\
      学号: & \underline\stuid   \\
      日期: & \underline\newdate \\
%      地点: & \underline\loc
    \end{tabular*}
  \end{subfigure}
\end{figure}

\begin{table}[!htbp]
  \centering
  \begin{tabular*}{0.7\linewidth}{llllll}
    课程名称: & \underline\course   & 指导老师: & \underline\tutor   & 成绩:       &  \underline\grades \\
    实验名称: & \underline\newtitle & 同组学生姓名:& \underline\group
  \end{tabular*}
\end{table}

% =============================================
% Part 2 Main document
% =============================================

\section{预习部分}
\subsection{实验目的}

\subsection{实验基本原理}

\subsection{主要仪器设备(含必要的元器件、工具)}

\section{实验操作部分}

  \subsection{实验数据、表格及数据处理}

    给出如下差分方程:
    $$y(n) - (0.5+a)\times y(n-1) + 0.5ay(n-2) = x(n)$$
    \begin{clause}
      \item 求解系统传输函数表达式。
      \item 当a取0.8,0.9,1.0,1.1时,画出零极点分布图。
      \item 根据(2)中a的取值,分别画出幅频响应函数。
    \end{clause}

  \subsection{实验操作部分}
  \subsubsection{自回归模型}
    \begin{clause}
      \item 导入数据"data.mat";
      \item 划分为训练集和测试集;
      \item 对其进行最小二乘法拟合;
      \item 初步预测;
      \item 用该预测项计算噪声项;
      \item 加入噪声项,优化AR模型;
      \item 得到最终模型;
      \item 预测,与测试集求残差;
      \item 观察结果;
    \end{clause}
  \subsection{结论}
  \subsubsection{自回归模型}
  \

\section{实验效果分析(包括仪器设备等使用效果)}




\section{教师评语}




\end{document}
